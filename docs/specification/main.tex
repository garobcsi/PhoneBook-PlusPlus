\documentclass[11pt, letterpaper, openany, oneside]{book}

\usepackage[utf8]{inputenc}
\usepackage{amsmath}
\usepackage{amsfonts}
\usepackage{amssymb}
\usepackage{graphicx}
\usepackage{sectsty}
\usepackage{titlesec}
\usepackage{color}
\usepackage{hyperref}
\usepackage{t1enc}
\usepackage{paralist}
\usepackage[hungarian]{babel}


\title{\textbf{Telefonykönv Specifikáció}}
\author{\LARGE Gáspár Róbert \vspace{4px} \\ K8FD5S}
\date{\today}

\hypersetup{
    colorlinks,
    linktoc=all,
    linkcolor={blue},
    citecolor={blue},
    urlcolor={blue}
}

\begin{document}

\maketitle

\tableofcontents

\chapter{Feladat}

\section{Célkitűzés}

A program C++ nyelven íródjon meg, és Linux operációs rendszeren fordítható legyen. A program legyen modularizált, azaz külön modulokra bontottan készüljön el, hogy ezeket külön-külön is használni lehessen. A modulok legyenek újrafelhasználhatók más projektjeinkben vagy más fejlesztők által írt programokban. A programkód legyen jól strukturált és dokumentált, hogy könnyen megérthető legyen más fejlesztők számára is.

\section{Funkcionális követelmények}

A moduloknak legyenek Unit tesztjei, amelyek a modulok működését leteszteli. A programnak legyen egy egyszerű, könnyen kezelhető felhasználói felülete, amelyen keresztül a Telefonykönv működését demonstrálhatjuk. Ez egy müködö program, amelyben a fejlesztők a telefonkönyv funkcionalitását tudják leteszteli terminálos alkalmazásként.

\section{Tiltások a funkcionális követelményhez}
A megoldáshoz ne használjon STL tárolót:

Az adatok tárolásához és kezeléséhez ne használjon STL tárolókat, helyette saját adatstruktúrákat és kezelőmetódusokat kell létrehozni.

\chapter{Feladatspecifikáció}

\section{Telefonykönv tartakma}
A telefonkönyvben kezdetben az alábbi adatokat akarjuk tárolni:
\vspace{5px}
\begin{compactitem}
    \item Név (vezetéknév, keresztnév)
    \item Becenév
    \item Cím
    \item Munkahelyi szám
    \item Privát szám
\end{compactitem}

\section{Telefonykönv műveletek}
Az alkalmazással minimum a következő műveleteket kívánjuk elvégezni (CRUD):

\vspace{5px}
\begin{compactitem}
    \item Adatok felvétele
    \item Adatok módosítása
    \item Adatok törlése
    \item Listázás
\end{compactitem}

\section{Keresési lehetőségek}
A legyen lehetőség keresni a telefonkönyvben szereplő névjegyek között. 

\newpage

\section{Importálás és exportálás}
A telefonkönyv modulnak támogatnia kell az adatok importálását és exportálását, például CSV formátumban. A modulnak ebböl a fileból kell betöltenie a névjegyeket, és ezt a filet kell frissítenie a módosítások után.

\section{Felhasználói felület}
A felhasználói felület csak a fejlesztők számára elérhető és használható, kivéve speciális esetekben, például tesztelés vagy demonstráció céljából.

\section{Adatok harmadik féllel való megosztása}
Az alkalmazásnak tilos az adatok harmadik féllel való megosztása vagy külső szolgáltatásokhoz való kapcsolódása az adatvédelem vagy biztonság megőrzése érdekében.

\end{document}