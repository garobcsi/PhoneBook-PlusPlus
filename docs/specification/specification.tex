\documentclass[11pt, letterpaper, openany, oneside]{book}

\usepackage[utf8]{inputenc}
\usepackage{amsmath}
\usepackage{amsfonts}
\usepackage{amssymb}
\usepackage{graphicx}
\usepackage{sectsty}
\usepackage{titlesec}
\usepackage{color}
\usepackage{hyperref}
\usepackage{t1enc}
\usepackage{paralist}
\usepackage[hungarian]{babel}

\title{\textbf{Telefonykönv Specifikáció}}
\author{\LARGE Gáspár Róbert \vspace{4px} \\ K8FD5S}
\date{\today}

\hypersetup{
    colorlinks,
    linktoc=all,
    linkcolor={blue},
    citecolor={blue},
    urlcolor={blue}
}

\begin{document}

\maketitle

\tableofcontents

\chapter{Feladat}

\section{Célkitűzés}

A program C++ nyelven íródjon meg, és Linux operációs rendszeren fordítható legyen. A program legyen modularizált, azaz külön modulokra bontottan készüljön el, hogy ezeket külön-külön is használni lehessen. A modulok legyenek újrafelhasználhatók más projektjeinkben vagy más fejlesztők által írt programokban. A programkód legyen jól strukturált és dokumentált, hogy könnyen megérthető legyen más fejlesztők számára is.

\section{Funkcionális követelmények}

A moduloknak legyenek Unit tesztjei, amelyek a modulok működését letesztelik. A programnak legyen egy egyszerű, könnyen kezelhető felhasználói felülete, amelyen keresztül a Telefonykönv működését demonstrálhatjuk. Ez egy müködő program, amelyben a fejlesztők a telefonkönyv funkcionalitását tudják leteszteli terminálos alkalmazásként.

\section{Tiltások a funkcionális követelményhez}

Az adatok tárolásához és kezeléséhez ne használjon STL tárolókat, helyette saját adatstruktúrákat és kezelőmetódusokat kell létrehozni.

\chapter{Feladatspecifikáció}

\section{Telefonykönv tartalma}
A telefonkönyvben az alábbi adatokat tároljuk:
\vspace{5px}
\begin{compactitem}
    \item Név (vezetéknév, keresztnév)
    \item Becenév
    \item Cím
    \item Munkahelyi szám
    \item Privát szám
\end{compactitem}

\section{Telefonykönv műveletek}
Az alkalmazással minimum a következő műveleteket végezhetjük el (CRUD):

\vspace{5px}
\begin{compactitem}
    \item Adatok felvétele
    \item Adatok módosítása
    \item Adatok törlése
    \item Listázás
\end{compactitem}

\section{Keresési lehetőségek}
Keresési lehetőséget biztosítunk a telefonkönyvben szereplő névjegyek között.

\newpage

\section{Importálás és exportálás}
A telefonkönyv modulnak támogatnia kell az adatok importálását és exportálását, például CSV formátumban. A modulnak ebből a fájlból kell betöltenie a névjegyeket, és ezt a fájlt kell frissítenie a módosítások után.

\section{Felhasználói felület}
A terminálos felhasználói felületet a fejlesztőknek kell implementálni, amit csak a fejlesztők használhatnak, kivéve speciális eseteket például demonstráció céljából. Az általános felhasználói felületet az átvevők fogják implementálni, amelyet a tényleges felhasználók használnak majd.

\end{document}